\subsection{Discrete Ordinate Method}
\begin{frame}
	\frametitle{Conventional Discrete Ordinate Method}
	When applying conventional DOM to either classical or semiclassical Boltzmann-BGK formulation, we can always render it as set of linear PDE's with fixed/constant advection velocity $\vec{c}_\sigma$,
	\begin{equation}
	\frac{\partial{f_\sigma}}{\partial{t}} +
	\vec{c_\sigma}\bullet\frac{\partial{f_\sigma}}{\partial{\vec{x}}} +
	\vec{F}\bullet\frac{\partial{f_\sigma}}{\partial{\vec{c_\sigma}}} = 
	-\frac{1}{\tau}(f_\sigma-f^{eq}_\sigma)
	\end{equation}
Here $\sigma$ $(\sigma = 1-N_v)$ stands as the index the number of discrete velocities. 
\end{frame}

\begin{frame}
We can also do the same for the semi-classical equilibrium distribution function. DOM renderes as a set of equations for the discrete values of $\vec{c}_\sigma$,
	\begin{equation}
	f_\sigma^{eq}=f^{eq}(\vec{x},\vec{c_\sigma},t)=\frac{1}{(\frac{1}{z})\exp\left({\frac{m}{2 k_B T}(\vec{c_\sigma}-\vec{u})^2}\right)+\theta}
	\end{equation}
\end{frame}

\begin{frame}
	\frametitle{Moment of the semi-classical distribution function}
	and the four moments integrals can be written in quadrature form, 
	\begin{eqnarray}
	\int f_\sigma d^3 c &=& \sum_\sigma W_\sigma \exp{(c_\sigma^2)} f(\vec{x},\vec{c_\sigma},t) = \rho \\
	\int \vec{c_\sigma} f_\sigma d^3 c &=& \sum_\sigma \vec{c_\sigma} W_\sigma \exp{(c_\sigma^2)} f(\vec{x},\vec{c_\sigma},t) = \rho \vec{u} \nonumber \\
	\int \frac{\vec{c_\sigma}^2}{2} f_\sigma d^3 c &=& \sum_\sigma \vec{c_\sigma}^2 W_\sigma \exp{(c_\sigma^2)} f(\vec{x},\vec{c_\sigma},t) = \rho E \nonumber \\
	\int \frac{(\vec{c_\sigma}-\vec{u})^2}{2} f_\sigma d^3 c &=& \sum_\sigma (\vec{c_\sigma}-\vec{u})^2 W_\sigma \exp{(c_\sigma^2)} f(\vec{x},\vec{c_\sigma},t) = \rho e \nonumber
	\end{eqnarray}
	Here we let $\vec{c}_\sigma$ be the abscissas or our quadrature and $W_\sigma$ their corresponding weigthing values. Here Gauss Hermite quadrature is used due to gaussians-like behavior of the three statistics.
\end{frame} 