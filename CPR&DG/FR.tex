\section{Flux Reconstruction in one-dimension}

%\begin{frame} \frametitle{Target Equation}
%	Let us consider the following hyperbolic conservation law of the form
%	\begin{equation}
% 	\left.\begin{aligned}
%        	&u_t + \div{f(u)} = 0,\\
%        	&u(0,\v{x}) = u_0(\v{x}),
%     		\end{aligned}
% 	\right\}
% 	\label{eq:conservation_law}
%	\end{equation}
%	for $\v{x} \in \mathcal{R}^3 $ and $t\geqslant 0$. \\
%	Here $u_0$ the initial condition given at $t=0$.
%\end{frame}

\begin{frame} \frametitle{Conservation equation in 1d}
	Let us consider the a one-dimensional advection equation in conservation form,
	\begin{equation}
	\left.\begin{aligned}
		&\pd{u}{t}+\pd{f(u)}{x}=0,\\
		&u(x,0) = u_0(x),
		\end{aligned}
	\right\}
	\label{eq:conservation_law_1d} 
	\end{equation}
	here $u(x,t)$ represents our conservation variable that is defiend for any $x \in [a,b]$ and $t\geqslant 0$. $u_0$ the initial condition given at $t=0$.  
\end{frame}

\subsection{Domain Discretization}

\begin{frame} \frametitle{Domain discretization}
	Having confined our spatial domain to $\forall x \in [a,b]$, let now divid it to $N$ intervals, where the $j$-th interval is defined,
	\begin{equation}
	\mathcal{I}_j = [x_{j-1/2},x_{j+1/2}], \;\;\text{ for } j = 1,\dots,N
	\end{equation}
	Each interval is treated as an standard element, $\mathcal{I} = [-1,1]$, with $\xi$ to varing inside $\mathcal{I}$ while $x$ varies inside $\mathcal{I}_j$. A mapping betten this two variables is defined as
	\begin{equation}
	x(\xi) = xc_j + \frac{\Delta x_j}{2} \xi,
	\label{eq:local2global_mapping}
	\end{equation}
	where $xc_j = \frac{1}{2}(x_{j+1/2}+x_{j-1/2})$ and $\Delta x_j = (x_{j+1/2}-x_{j-1/2})$, are the center and the with of $\mathcal{I}$. Moreover, if the elements are not uniform we would define $\Delta x = max_{j=1,\dots,N} \Delta x_j$. 
\end{frame}

%\begin{frame}
%	An important property solution of the scalar conservation laws is that it satisfies a strict maximun principle. e.i. for a one-dimensional law as in (\ref{eq:conservation_law_1d}) if 
%	\begin{align}
%	&M = max_x \ u_0(x), & m = min_x \ u_0(x),
%	\end{align}
%	Then $u(t,x) \in [m,M]$ for any $x$ and $t$. In particularly, the solution will not be %negative if $u(0,x)\geqslant 0$.
%\end{frame}
	
\begin{frame} 
	We now wish to approach a solution of $u(x,t)$ on the standard element $\mathcal{I}$ first by defining a set of solution points $\xi_k$, where $k = 1,\dots,K$. The later should be choosen in advance and must be
	\begin{itemize}
	\item optimum for a quadrature method on the standard element and
	\item optimum for a interpolation method.
	\end{itemize}
	In the following progress we concentrate on abscissas of:
	\begin{itemize}
	\item Gauss Legendre (Legendre)
	\item Legendre-Gauss-Lobatto (LGL)
	\end{itemize}
	having specified the type and number of solution points, we use (\ref{eq:local2global_mapping}) to formulate the global coordiantes of the solutions points,
	\begin{equation}
	x_{j,k} = xc_j + \frac{\Delta x_j}{2} \xi_k,
	\end{equation}
\end{frame}

\subsection{Numerical approximation of the solution}

\begin{frame}
	For every element $I_j$ of our domain, the solution is defined as $u_{j,k}^n$ for, $j=1,\dots,N$, $k = 1,\dots,K$ and every time step level $n$. Using the standard element approach, the solution for the $j$-th element is given by,
	\begin{equation}
	u_{j}(\xi) = \sum_{k=1}^{K} u_{j,k} \mathcal{L}_k(\xi)
	\label{eq:u_interpolation_in_I}
	\end{equation}
	where
	\begin{equation}
	\mathcal{L}_k(\xi) = \prod_{i=1\;i\neq k}^{K} \frac{(\xi-\xi_i)}{(\xi_k-\xi_i)}
	\label{eq:lagrangepolynomial}
	\end{equation}
	where (\ref{eq:lagrangepolynomial}) is a Lagrange inperpolatory polynomial. Notice that $u_j(\xi)$ is then degree $K-1$.
\end{frame}

\begin{frame} \frametitle{Approximating $u_{j}$ values at the boundaries}
	Using definition (\ref{eq:u_interpolation_in_I}), we can reconstruct the boundary values by doing,
	\begin{align}
	&u_{j-1/2}^{+}=u_j(-1)& &\text{ and }& &u_{j+1/2}^{-}=u_j(1)&
	\end{align}
	if we choosen solution points coincide with the limits of our elements, then boundary values are simply defined by,
	\begin{align}
	&u_{j-1/2}^{+}=u_{j,1}& &\text{ and }& &u_{j+1/2}^{-}=u_{j,K}&
	\end{align}
\end{frame}

\begin{frame}
	Similarly we would also requiere to compute the fluxes at K solution points. This can be easy acomplish by doing,
	\begin{equation}
	f_{j,k} = f(u_{j,k}),
	\end{equation}
	and we interpolate flux at the boundaries by using the standar element,
	\begin{equation}
	f_j(\xi) = \sum_{k=1}^{K} f_{j,k} \mathcal{L}_k(\xi)
	\label{eq:f_interpolation_in_I}
	\end{equation}
	Also notice that $f_j(\xi)$ is a polynomial of order $K-1$.
\end{frame}

\begin{frame}
	By we compare the elements adjacent boundary/face values, is easy to notice that {\only<2>{\color{blue}}$f_j(1)\neq f_{j+1}(-1)$}.
	
	Therefore we must define a continuous flux function to deal with the data of adjacents cells.
\end{frame}

\begin{frame}
	However, this flux function, $F$, must meet the following requirements
	\begin{itemize}
	\item Must use data from the boundaries, i.e. 
	 \[ F_j(-1)= \hat{f}_{j-1/2} \text{ and } F_j(1)= \hat{f}_{j+1/2}.\] 
	Where \[ \hat{f}_{j+1/2} = \hat{f}(f_{j+1/2}^{-},f_{j+1/2}^{+}) \]
	is a Riemann flux. \pause
	\item In each cell $\mathcal{I}_j$ the degree of $F_j(\xi)$ should be $K$.
	\[ \text{i.e., one degree higher than the solution polynomial.}\] \pause
	\item $F_j(\xi)$ is required to approximate the discontinuous solution $f_j(\xi)$. 
	i.e., \[ F_j(\xi)-f_j(\xi) \approx 0.\] \pause
	\end{itemize}
\end{frame}

\begin{frame} \frametitle{The continuous Flux function}
	Let us define the flux function as:
	\begin{equation}
	F_j(\xi) = {\only<4>{\color{red}}f_j(\xi)}+ 
	{\only<2>{\color{blue}} \left[ \hat{f}_{j-1/2} - f_j(-1) \right]} g_{LB}(\xi) +
	{\only<3>{\color{green}}\left[ \hat{f}_{j+1/2} - f_j( 1) \right]} g_{RB}(\xi),
	\;\; \xi \in [-1,1]
	\label{eq:continuous_flux_func}
	\end{equation}
	Here $g_LB$ is the correction function at the left boundary satisfying,
	\begin{align}
	&g_{LB}(-1)=1& &\text{ and }&  &g_{LB}(1)=0&
	\end{align}
	and similarly $g_RB$ is the correction function at the right boundary and satisfies, 
	\begin{align}
	&g_{RB}(-1)=0& &\text{ and }&  &g_{RB}(1)=1&
	\end{align}
\end{frame}

\begin{frame}
	The derivate of $F_j(\xi)$ evaluated at the solution points $\xi_k$ for $k=1,\dots,K$, is then
	\begin{equation}
	\pd{F_j(\xi_k)}{\xi} = \pd{f_j(\xi_k)}{\xi}+ 
	\left[ \hat{f}_{j-1/2} - f_j(-1) \right] g_{LB}^{\prime}(\xi) +
	\left[ \hat{f}_{j+1/2} - f_j( 1) \right] g_{RB}^{\prime}(\xi),
	\;\; \xi \in [-1,1]
	\label{eq:continuous_flux_func}
	\end{equation}		
	Using the mapping defined in equation (\ref{eq:local2global_mapping}) and knowing that the derivate if $F_j(x)$ at the solution points $x_{j,k}$ is,
	\begin{equation}
	(F_j)_x (x_{j,k}) = \frac{2}{\Delta x_j} (F_j)_{\xi}(\xi_k)
	\end{equation}
Finally we update the numerical solution using the following ODE,
	\begin{equation}
	\pd{u_{j,k}}{t}+\frac{2}{\Delta x_j}(F_j)_\xi (\xi_k)=0
	\end{equation}
\end{frame}

\subsection{Correction functions}

\begin{frame} \frametitle{The Radau Polynomials}
\begin{itemize}
	\item Huynh in \cite{Huynh2007} introduces and tested many correction functions candidates.
	\item Here, we are more interested in the radau polynomails. By using Radau polynomials it is possible to demostrate that the resulting scheme recovers a DG scheme for hyperbolic equations laws; at least in the linear case.
	\item The Radau polynomials are of two kinds, left and right; and can be expressed in terms of Legendre Polyonomials, $P_K$, as
\end{itemize}
	\begin{align}
		R_{Left,K}  =& \frac{1}{2}(P_K+P_{K-1}), \\
		R_{Right,K} =& \frac{(-1)^K}{2}(P_K-P_{K-1})
	\end{align}
\end{frame}

\begin{frame} \frametitle{Right Radau Polynomials}
	Let us write explicitly the first four right Radau polynomials,
\begin{figure}
	\centering
	\includegraphics[width=0.8\textwidth]{images/RadauRight.png} 
	\label{fig:RadauRight}
	\caption{Right Radau Polynomials}
\end{figure}	
	In the above figure notice that the main characteristic of this polynomial is
\begin{align}
	&R_{Right,K}(-1) = 1,& &and& &R_{Right,K}(1) = 0.& 
\end{align}
\end{frame}


\begin{frame} \frametitle{Left Radau Polynomials}
	Let us write explicitly the first four left Radau polynomials,
\begin{figure}
	\centering
	\includegraphics[width=0.8\textwidth]{images/RadauLeft.png}
	\label{fig:RadauLeft}
	\caption{Left Radau Polynomials}
\end{figure}
	Similarly, we notice that the behavior of this polynomials is mirror symmetric,
\begin{align}
	&R_{Left,K}(-1) = 0,& &and& &R_{Left,K}(1) = 1.& 
\end{align}
\end{frame}

\begin{frame} \frametitle{The correction functions}
	Notice that when calling in our correction functions, $g_{LB} \& g_{RB}$, into the continuous flux function, $F_j(\xi)$, we must make sure that
	\begin{align}
	&g_{LB} = R_{Rright,K}& &and& &g_{RB} = R_{Left,K}&
	\end{align}
\end{frame}

\begin{frame} \frametitle{Remarks}
	\begin{itemize}
	\item Recall that the $g_{LB}$ and $g_{RB}$ have not been uniquely defined, more can be found in Huynh's work \cite{Huynh2007}.
	\item Because of their symmetry we only considere $g_{LB}$, and we would be only refered as $g$. 
	\item Notice, however that $g'(\xi)$ is required to be meet the accuaracy of $u_j{\xi}$ and $f_{\xi}$ of degree $K-1$. Therefore g has to be choosen of degree $K$.  
	\item Huynh in \cite{Huynh2007} by choosing Radau Polynomials as the correction functions, demostrated that CPR framework recovers a DG method, at least for linear case.
	\end{itemize}
\end{frame}
