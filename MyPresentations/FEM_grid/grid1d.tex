\begin{frame} \frametitle{One-dimensional Mesh}
	Let us considere a simple scenario in one-dimension
	\begin{figure}
		\centering
		\begin{subfigure}[b]{0.70\textwidth}
			\centering
			\includegraphics[trim = 0mm 0mm 0mm 0mm,clip,width=\textwidth]{../FEM_grid/Mesh1d_pics/mesh1d}
			\caption{1D Mesh}
			\label{fig:uniform_mesh_1d}
		\end{subfigure}%
	\end{figure}
	let us set $x \in [a,b] = [0,1]$ and divide the domain into $n = 10$ elements/cells.
\end{frame}

\begin{frame} \frametitle{Vertex and EtoV}
	Then we can build a simple node list and a vectex connection matrix
	\begin{figure}
        \centering
        \begin{subfigure}[b]{0.20\textwidth}
                \centering
                \includegraphics[trim = 0mm 0mm 0mm 0mm,clip,width=\textwidth]{../FEM_grid/Mesh1d_pics/VX}
                \caption{Vertex nodes}
                \label{fig:VX}
        \end{subfigure}%
				~ %add desired spacing between images, e. g. ~, \quad, \qquad etc.
          %(or a blank line to force the subfigure onto a new line)
        \begin{subfigure}[b]{0.70\textwidth}
                \centering
                \includegraphics[trim = 0mm 0mm 0mm 0mm,clip,width=\textwidth]{../FEM_grid/Mesh1d_pics/EtoV}
                \caption{Elements to Vertex}
                \label{fig:EtoV}
				\end{subfigure}
				%\caption{Typical Application for the Classical Boltzmann Equation}
				\label{fig:Nodal Grid}
	\end{figure}
	Note that here I'll focus only on the vertex (nodes at edges of my element)
\end{frame}

\begin{frame} \frametitle{FtoV}
	following ideas from \cite{Karup2012}, we can further explore a way to build and FtoV matrix. 
	In our example this array has the form,
	\begin{figure}
		\centering
		\begin{subfigure}[b]{0.70\textwidth}
			\centering
			\includegraphics[trim = 0mm 0mm 0mm 0mm,clip,width=\textwidth]{../FEM_grid/Mesh1d_pics/FtoV}
			\caption{Face to elements Vertex}
			\label{fig:FtoV}
		\end{subfigure}%
	\end{figure}
	Notice that the use of a sparce structure can be very beneficial in order to save time and computer resources.
\end{frame}

\begin{frame} \frametitle{FtoF}
	Now we can recover more information of the Face to Face connection by,
	\begin{equation}
		FtoF = (FtoV)(FtoV)^T - I.
	\end{equation}
	For our example we get,
	\begin{figure}
		\centering
		\begin{subfigure}[b]{0.85\textwidth}
			\centering
			\includegraphics[trim = 0mm 0mm 0mm 0mm,clip,width=\textwidth]{../FEM_grid/Mesh1d_pics/FtoF}
			\caption{Face to Face matrix}
			\label{fig:FtoF}
		\end{subfigure}%
	\end{figure}
	Form the avobe result we identify by the element row-column indexes the information of Element and global face conection.
\end{frame}

\begin{frame} \frametitle{Local node list}
	Given the one-dimensional element, set the label de local faces as $[1,2]$, for the left and right faces respectively.
	We can transform the global face index to local face index by doing,
	\begin{figure}
		\centering
		\begin{subfigure}[b]{0.50\textwidth}
			\centering
			\includegraphics[trim = 0mm 0mm 0mm 0mm,clip,width=\textwidth]{../FEM_grid/Mesh1d_pics/FtoE_local}
			\caption{Computing local element and face from FtoF matrix}
			\label{fig:FtoF_local}
		\end{subfigure}%
	\end{figure}
\end{frame}

\begin{frame} \frametitle{EtoF \& EtoE}
	Finally with the local indexes computed with the strategy of Fig. [\ref{fig:FtoF_local}], we can obtain
	\begin{figure}
        \centering
        \begin{subfigure}[b]{0.20\textwidth}
                \centering
                \includegraphics[trim = 1mm 0mm 0mm 0mm,clip,width=\textwidth]{../FEM_grid/Mesh1d_pics/EtoE}
                \caption{Element to Element connectivity}
                \label{fig:EtoE}
        \end{subfigure}%
				~ %add desired spacing between images, e. g. ~, \quad, \qquad etc.
          %(or a blank line to force the subfigure onto a new line)
        \begin{subfigure}[b]{0.20\textwidth}
                \centering
                \includegraphics[trim = 1mm 0mm 0mm 0mm,clip,width=\textwidth]{../FEM_grid/Mesh1d_pics/EtoF}
                \caption{Elements to Face connectivity}
                \label{fig:EtoF}
				\end{subfigure}
				%\caption{Typical Application for the Classical Boltzmann Equation}
				\label{fig:EtoE_and_EtoF}
	\end{figure}
	Now that we have this relations, we are out ot create boundary maps for faces inside the domain and outside the domain.
\end{frame}
