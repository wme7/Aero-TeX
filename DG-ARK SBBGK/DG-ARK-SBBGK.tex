\documentclass[a4paper,10pt]{article}
\usepackage[utf8]{inputenc}

\title{A Direct Discontinuous Galerkin Additive Runge-Kutta Solver 
for Semiclassical Botlzmann Equation of Arbitrary Statistics}
\author{Manuel Diaz}

\begin{document}

\maketitle

\begin{abstract}
We report in this paper our most recent progress on implementing and analyzing a class of Discontinuous Galerkin method with the main objetice of solving directly the Semiclassical Boltzmann Equation with BGK appproximation using the Additive Runge Kutta (ARK) time integration. The later time integration method is choosen to deal with the BGK approximation of the collision operator. The BGK source imposes in our hyperbolic PDE a highly stiffed term which limits the time step we can use. The explicit Runge-Kutta (ERK) has been historically the a strong part of the DG original formulation, but for our case it won't fulfill the flexibility desired to make our Direct solver fast and computationally efficient in dealing with more complex application in future works.

\end{abstract}

\section{Introduction}
First we consider a numerical solution for the hyperbolic conservation law
\begin{equation}
 u_t + \nabla_{x_i} \cdot f(u) = 0
\end{equation}
equipped with suitable initial and boundary conditions. Here
$ u = (u_1, \dots , u_m)^t$ and $x = (x_1, \dots , x_d)$. 
We would concentrate first on cases where d = m = 1. We use this as a model and framework of the scheme, to prove theoretical results and highly essential ingredients in the method, while keeping in mind that these methods are naturally extendable to $d > 1$ and or $m > 1$.
We are going to implement Discontinuous Galerkin Method (DGM). Which is based on polynomial reconstruction ideas of classical Finite Element Method (FEM) and flux conservation principles of the Finite Volume Method (FVM).

\section{The Method}
Discontinuous Galerkin Method was introduced in 1973 by Reed and Hill [ref?], in the framework of neutron transport (steady state linear hyperbolic equations). Major contributions have been carried out by Cockburn et al. in series or papers [refs?], which stablished a initial framework for all the new kinds of methods for solving non-linear hyperbolic time dependent conservation laws.

In this paper we start the investigation of adapting DGM to deal with our Direct Solver by studying the numerical properties of solving an nonhomogenous scalar advection equation.  
\begin{eqnarray}
 u_t + \nabla_x \cdot f(u) = S(u) \\
 u(x,0) = u_0(x) \nonumber
\end{eqnarray}
A semidiscrete Discontinuous Galerkin method for solving the non-homogenous hyperbolic PDE (2) is usually obtained by multiplying (2) by a test function $v_h \in V_h$, integrating over suitable domains, and finally integrating by parts. If the space of trial functions $U_h$ and hte one of the test functions $V_h$ coincide, the method is called a Galerkin Method, otherwise it is called a Petrov-Galerkin method.

Let points $x_i$ be the center of the cells $I_i = [x_{i-1/2},x_{i+1/2}]$, and we denote the cell sizes by $\Delta x_i = x_{i-1/2}-x_{i+1/2}$ and the maximun cell size by $h =  sup_i(\Delta x_i)$ The Finite Element Method we are going to use is Discontinuous Galerkin, that is for the solution as well as the test function is given by
\begin{equation}
 U_h = V_h = V_h^k = \{p : p|_{I_i} \in P^k(I_i)\},
\end{equation}
where $p_k(I_i)$ is the space of polynomials of degree $\leq k$ on the $I_i$. Note that in $V_h^k$, the functions are allowed to have jumps at the interfaces $x_{i+1/2}$. Is this characteristic of the method that make it different to any finite element method and render it's characteristic of explicit in semidiscrete form with is suitable for any explicit time discretization method.

We adopt a local Orthogonal basis over $I_i$, $\{v_l^{(i)}(x), l=0,1,2, \dots, k\}$ namely the Legendre polynomials.

The numerical solution $u^h(x,t)$ in the space $V_h^k$ can be written as
\begin{equation}
 u^h(x,t) = \sum_{l=0}^{k} u_i^{(l)}(t)v_l^({i})(x), \; \; for \; x \in I_i
\end{equation}
and the Legendre Coeficients or degrees of freedom $u_i^{(l)}(t)$ are the moments defined by
\begin{equation}
 u_i^{(l)}(t) = 1/a_l \int_{I_i} u^h(x,t)v_l^{(i)}(x),dx, \; \; l = 0,1, \cdots,k
\end{equation}
where $a_l = \int_{I_i} (v_l^{(i)}(x))^2 dx$ are the normalization constants since the basis is not orthonormal. In order to determine the approximate solution, what we would like to do is to evolve the degress of freedom $u_i^{(l)}$. To do this we first we multiply (2) by $v_h \in V_h^k$, and integrate over $I_i$ and replace the exact solution $u$ by its approximation $u^h$

\begin{eqnarray}
 d/dt u_i^{(l)} + 1/a_l(-\int_{I_i} f(u^h(x,t))d/dx v^{(i)}x dx \\ \nonumber
 	+ f(u_{i+1/2}^{-},u_{i+1/2}^{+})v_l^{(i)}(x_{i+1/2}) 
 	- f(u_{i-1/2}^{-},u_{i+1/2}^{-})v_l^{(i)}(x_{i+1/2}) \\ \nonumber
	= 0, \; \;  l = 0,1, \cdots ,k
\end{eqnarray}

if we write our the first equation in (6),
\begin{equation}
 d/dt u_i^{(l)} + 1/{\Delta x}f(u_{i+1/2}^{-},u_{i+1/2}^{+})
 	- f(u_{i-1/2}^{-},u_{i+1/2}^{-})
\end{equation}

\section{The Implementation}
Here is how we plan to solve the SBBGK using the our DG-ARK

\end{document}
