\documentclass[a4paper]{report}
% Coded by Manuel Diaz, NTU, Feb.2013

% Packages
\usepackage[utf8]{inputenc}
\usepackage{amsmath}
\usepackage{amsfonts}
\usepackage{amssymb}

% Uncomment to test compilation of single files:
%\includeonly{file1,file2}

% Modify '\vec{}' & '\hat{}' commands to generate bold vector characters 
\let\oldhat\hat
\renewcommand{\vec}[1]{\mathbf{#1}}
\renewcommand{\hat}[1]{\oldhat{\mathbf{#1}}}

% Star Document
\begin{document}

% Make title page
\author{Manuel Diaz}
\title{My PhD Thesis}
\date{February 2012}
\maketitle

% Roman Numeration for the beginning of the document,
\pagenumbering{roman}
\tableofcontents
%\listoffigures
%\listoftables

% Acknowledgements
\chapter*{Acknowledgements}
I wish to thank all my dear friends for their support, my professor for his coaching, and the creator for its pattience towards myself.

% Abstract
\begin{abstract}
During the recent years, Discontinuous Galerkin Methods (DGM) have become widely popular and have attracted the attention of many CFD and numerical practitioners due to its great accuaracy and the parallel advangates that the method provides to solve partial differential equations. DG methods takes the best of two worlds, i.e. the simplicity and fundamentals conservative principles of finite volume formulations and the domain flexibility of spectral methods such us finite element methods. Is in this context that new practitioners like the present author starts its own journey in learning the method's formulation, numerical implementation and formulation for practical applications. 
\end{abstract}

% Chapters numbering
\pagenumbering{arabic}

% Ideal Quatum Gas Dynamics
%\documentclass[11pt,fleqn]{article}
%\begin{document}

\chapter{Idea Quantum Gas Dynamics Chapter}

This is the theory behind the Semi-classical Boltzmann Equation

%\end{document}


% Semi-classical Botlzmann BGK
%\documentclass[11pt,fleqn]{article}
%\begin{document}

\chapter{Semiclassical Boltzmann Equation with BGK collision collision Operator}

Now with the basic theory we are know ready to formulate a parrallel way to model transpor phenomena and their carriers as if they where a classial or quantum gas.


%\end{document}


% Numerical Methods for Conservation Laws
	% Hyperbolic Conservation Laws
	%\documentclass[11pt,fleqn]{article}
%\begin{document}

\chapter{Brief Introduction to Conservation Laws}
In this chapter I would introduce the most fundamental concepts of conservation laws to up show the classical solution of invicid and classical burgers equation.

%\end{document}

	% Upwind
	%\documentclass[11pt,fleqn]{article}
%\begin{document}

\chapter{Upwind Method}
This is the most fundamental an basic chapter for any beginer in conservation laws.

%\end{document}

	% TVD
	%\documentclass[11pt,fleqn]{article}
%\begin{document}

\chapter{TVD Method}
This is the way to apply TVD Method

%\end{document}

	% WENO
	%\documentclass[11pt,fleqn]{article}
%\begin{document}

\chapter{WENO Method}
this is the way to apply weno method
done by 

%\end{document}

	% DGM
	\include{DGM}

% Bibliography
\bibliography{DGM_refs}
\bibliographystyle{unsrt}

\end{document}
