\subsection{Velocity Space Discretization}
\begin{frame} \frametitle{Discrete Ordinate Method DOM}
	DOM is then applied set fixed/discrete velocities ($v_\sigma,v_\delta$) in velocity space. That is the value $f_{\sigma,\delta}^{M}(t,x,y) = f(t,x,y,v_\sigma,v_\delta)$ and (\ref{eq:2d_boltzmannBGK}) in phace space is reduced to a set of hyperbolic partial differential equations with source term in the physical space.
	\begin{align}
		\frac{\partial{f_{\sigma,\delta}}}{\partial{t}} +
		v_{\sigma}\cdot\frac{\partial{f_{\sigma,\delta}}}{\partial x} +
		v_{\delta}\cdot\frac{\partial{f_{\sigma,\delta}}}{\partial y} = 
		-\frac{1}{\tau}(f_{\sigma,\delta}-f^{eq}_{\sigma,\delta})
	\end{align}
Here $\sigma$ $(\sigma = 1,\dots,N_{1})$ and $\delta$ $(\delta = 1,\dots,N_{2})$ are the indexes of each discrete velocity. Each equation has a corresponding equilibrium distribution funcion, namely
	\begin{align}
		f_{\sigma,\delta}^{eq}(t,x,y,v_\sigma,v_\delta) = 
		\frac{\rho}{(\pi T)^{\frac{1}{2}}} \exp\left({-\frac{(v_{\sigma}-u_x)^2+(v_{\delta}-u_y)^2}{T}} \right)
	\end{align}
\end{frame}

\begin{frame}
	\frametitle{Moments of the classical distribution function}
	We use Gauss-Hermite quadrature rule to re-express the integral moments of our distribution function,
	\begin{subequations}
		\begin{align}
		\rho = \int f_{\sigma,\delta} dv_x dv_y 
				&= \sum_{\sigma=1}^{N_1} \sum_{\delta=1}^{N_2} W_\sigma W_\delta \mathrm{e}^{v_\sigma^2} \mathrm{e}^{v_\delta^2} f_{\sigma,\delta}\\
		\rho u_x = \int v_\sigma f_{\sigma,\delta} dv_x dv_y 
				&= \sum_{\sigma=1}^{N_1} \sum_{\delta=1}^{N_2} v_\sigma W_\sigma W_\delta \mathrm{e}^{v_\sigma^2} \mathrm{e}^{v_\delta^2} f_{\sigma,\delta}\\
		\rho u_y = \int v_\delta f_{\sigma,\delta} dv_x dv_y 
				&= \sum_{\sigma=1}^{N_1} \sum_{\delta=1}^{N_2} v_\sigma W_\sigma W_\delta \mathrm{e}^{v_\sigma^2} \mathrm{e}^{v_\delta^2} f_{\sigma,\delta}\\
		E = \int \frac{v_\sigma^2+v_\delta^2}{2} f_{\sigma,\delta} dv_x dv_y 
				&= \sum_{\sigma=1}^{N_1} \sum_{\delta=1}^{N_2} \frac{v_\sigma^2+v_\delta^2}{2} W_\sigma W_\delta \mathrm{e}^{v_\sigma^2} \mathrm{e}^{v_\delta^2} f_{\sigma,\delta}\\
		\rho \epsilon = \int \frac{c_\sigma^2+c_\delta^2}{2} f_{\sigma,\delta} dv_x dv_y 
				&= \sum_{\sigma=1}^{N_1} \sum_{\delta=1}^{N_2} \frac{c_\sigma^2+c_\delta^2}{2} W_\sigma W_\delta \mathrm{e}^{v_\sigma^2} \mathrm{e}^{v_\delta^2} f_{\sigma,\delta}
		\end{align}
	\end{subequations}
	Here we let $v_\sigma$ be the abscissas of our quadrature and $W_\sigma$ their corresponding weigthing values. 
\end{frame} 