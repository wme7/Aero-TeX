\section{Motivation}

\begin{frame} 
  Let us remember what is mean free path in classical kinetic theory,
  \begin{figure}
        \centering
        \begin{subfigure}[b]{0.30\textwidth}
                \centering
                \includegraphics[trim = 0mm 0mm 0mm 0mm,clip,width=\textwidth]{../IWCSE_2013/Intro_pics/mean_free_path1}
                \caption{Density of particles \& Pressure}
                \label{fig:particles_pressure}
        \end{subfigure}%
        ~ %add desired spacing between images, e. g. ~, \quad, \qquad etc.
          %(or a blank line to force the subfigure onto a new line)
        \begin{subfigure}[b]{0.30\textwidth}
                \centering
                \includegraphics[trim = 1mm 1mm 1mm 1mm,clip,width=\textwidth]{../IWCSE_2013/Intro_pics/mean_free_path2}
                \caption{Mean free path}
                \label{fig:mean_free_path}
				\end{subfigure}
				~ %add desired spacing between images, e. g. ~, \quad, \qquad etc.
          %(or a blank line to force the subfigure onto a new line)
        \begin{subfigure}[b]{0.30\textwidth}
                \centering
                \includegraphics[trim = 0mm 0mm 0mm 0mm,clip,width=\textwidth]{../IWCSE_2013/Intro_pics/charac_length}
                \caption{Lenght Scale}
                \label{fig:lenght_scale}
				\end{subfigure}
				%\caption{Typical Application for the Classical Boltzmann Equation}
	\label{fig:MeanFreePath_KnudsenNumber}
 \end{figure}
	The knudsen number is the ratio of a characteristic lenght and the mean free path,
	\begin{equation*}
	Kn = \frac{L}{\lambda}
	\end{equation*}
\end{frame}

\begin{frame} \frametitle{Why we want to solve Boltzmann Equation?}
  Regimes of continuous and rarefied gases
  \begin{figure}
  \centering
  \includegraphics[height = 40 mm,width = 80 mm,trim=1pt 1pt 1pt 1pt]{../IWCSE_2013/Intro_pics/Knudsen_number_ranges}
  \caption{Ranges of applicability for methods with respect to Knudsen number / Dimensionless degree of rarefaction}
  \label{fig:Knudsen_number_ranges}
  \end{figure}
\end{frame}

\begin{frame}
 \begin{figure}
        \centering
        \begin{subfigure}[b]{0.30\textwidth}
                \centering
                \includegraphics[trim = 1mm 1mm 1mm 1mm,clip,width=\textwidth]{../IWCSE_2013/Intro_pics/micro_gas_flows}
                \caption{Micro-gas Flows}
                \label{fig:micro_gas_flows}
        \end{subfigure}%
        ~ %add desired spacing between images, e. g. ~, \quad, \qquad etc.
          %(or a blank line to force the subfigure onto a new line)
        \begin{subfigure}[b]{0.30\textwidth}
                \centering
                \includegraphics[trim = 1mm 1mm 1mm 1mm,clip,width=\textwidth]{../IWCSE_2013/Intro_pics/Artemis_satellite}
                \caption{Orbital motion}
                \label{fig:orbital_motion}
        \end{subfigure}
				~ %add desired spacing between images, e. g. ~, \quad, \qquad etc.
          %(or a blank line to force the subfigure onto a new line)
        \begin{subfigure}[b]{0.30\textwidth}
                \centering
                \includegraphics[trim = 1mm 1mm 1mm 1mm,clip,width=\textwidth]{../IWCSE_2013/Intro_pics/space_reentry}
                \caption{Space Reentry}
                \label{fig:space_reentry}
        \end{subfigure}
        \caption{Typical Application for the Classical Boltzmann Equation}
	\label{fig:Classical_Applications}
 \end{figure}
\end{frame}

\begin{frame}
 What if we use a different carrier? \\ : The Semi-classical Boltzmann Equation
  \begin{figure}
        \centering
        \begin{subfigure}[b]{0.30\textwidth}
                \centering
                \includegraphics[trim = 1mm 1mm 1mm 1mm,clip,width=\textwidth]{../IWCSE_2013/Motivation_pics/Electron_Flow}
                \caption{Electron Flows in Semiconductors}
                \label{fig:Electron_Flow}
        \end{subfigure}
        ~ %add desired spacing between images, e. g. ~, \quad, \qquad etc.
          %(or a blank line to force the subfigure onto a new line)
        \begin{subfigure}[b]{0.30\textwidth}
                \centering
                \includegraphics[trim = 1mm 1mm 1mm 1mm,clip,width=\textwidth]{../IWCSE_2013/Motivation_pics/Plasma-lamp}
                \caption{Plasma Modelation}
                \label{fig:Plasma_lamp}
        \end{subfigure}
        ~ %add desired spacing between images, e. g. ~, \quad, \qquad etc.
          %(or a blank line to force the subfigure onto a new line)
        \begin{subfigure}[b]{0.30\textwidth}
                \centering
                \includegraphics[trim = 1mm 1mm 1mm 1mm,clip,width=\textwidth]{../IWCSE_2013/Motivation_pics/HeatTransfer_compositeMaterials}
                \caption{Heat Transfer in composite materials}
                \label{fig:HeatTransfer_compositeMaterials}
        \end{subfigure}
        \caption{Typical Application for the Semi-classical Boltzmann Equation, Modelation of Quantum transport phenomena}
	\label{fig:Semiclassical_Applications}
 \end{figure}
\end{frame}