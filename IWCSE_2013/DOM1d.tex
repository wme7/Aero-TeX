\subsection{Velocity Space Discretization}
\begin{frame} \frametitle{Discrete Ordinate Method DOM}
	Following ideas in \cite{Yang1995323}, let us use DOM to render a set of linear PDE's with fixed/constant advection velocity $v_\sigma$,
	\begin{align}
		\frac{\partial{f_\sigma}}{\partial{t}} +
		v_\sigma\cdot\frac{\partial{f_\sigma}}{\partial x} = 
		-\frac{1}{\tau}(f_\sigma-f^{eq}_\sigma)
	\end{align}
Here $\sigma$ $(\sigma = 1,\dots,N_v)$ stands as the index the number of discrete velocities. Each equation has a corresponding equilibrium distribution funcion, namely
	\begin{align}
		f_\sigma^{eq}(t,x,v_\sigma)= \frac{\rho}{(\pi T)^{\frac{1}{2}}} \exp\left({-\frac{(v_\sigma-u)^2}{T}} \right)
	\end{align}
\end{frame}

\begin{frame}
	\frametitle{Moments of the classical distribution function}
	We use Gauss-Hermite quadrature rule to re-express the integral moments of our distribution function,
	\begin{subequations}
	\begin{align}
	\rho(t,x) = \int f_\sigma \ dv &= \sum_{\sigma=1}^{N_v} W_\sigma \mathrm{e}^{v_\sigma^2} f_\sigma\\
	\rho(t,x) u(t,x) = \int v_\sigma f_\sigma dv &= \sum_{\sigma=1}^{N_v} v_\sigma W_\sigma \mathrm{e}^{v_\sigma^2} f_\sigma\\
	E(t,x) = \int \frac{v_\sigma^2}{2} f_\sigma dv &= \sum_{\sigma=1}^{N_v} \frac{v_\sigma^2}{2} W_\sigma \mathrm{e}^{v_\sigma^2} f_\sigma\\
	\rho \epsilon(t,x) = \frac{3}{2}\rho RT(t,x) = \int \frac{C_\sigma^2}{2} f_\sigma dv &= \sum_{\sigma=1}^{N_v} \frac{c_\sigma^2}{2} W_\sigma \mathrm{e}^{v_\sigma^2} f_\sigma
	\end{align}
	\end{subequations}
	Here we let $v_\sigma$ be the abscissas of our quadrature and $W_\sigma$ their corresponding weigthing values. 
\end{frame} 

\begin{frame}
Graphical perspective of the Maxwellian distribution function for a typical Riemann IC in phase (velocity) space.
\begin{figure}
	\centering
	\includegraphics[trim = 20mm 10mm 30mm 20mm,clip,width=\textwidth]{../IWCSE_2013/BasicConcepts_pics/MB_IC}
	\caption{Sod's Shock tube initial condition}
	\label{fig:f_IC}
\end{figure}
\end{frame}

\begin{frame}
Graphical perspective of the evolution of the Maxwellian distribution function using Boltzmann-BGK model.
\begin{figure}
	\centering
	\includegraphics[trim = 20mm 10mm 30mm 20mm,clip,width=\textwidth]{../IWCSE_2013/BasicConcepts_pics/MB_evolution}
	\caption{Evolution by BBGK}
	\label{fig:f_evolution}
\end{figure}
\end{frame}