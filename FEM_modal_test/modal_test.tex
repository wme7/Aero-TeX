\documentclass[10pt,twoside,a4paper]{article}
\usepackage{amsmath}
\usepackage{amssymb}
\usepackage{amsfonts}
\usepackage{graphicx}

% to create figure arrays
%\RequirePackage{fix-cm}
\usepackage{subcaption}

% load package with ``framed'' and ``numbered'' option.
\usepackage[framed,numbered,autolinebreaks,useliterate]{mcode}

% Bold Vectors
\renewcommand{\vec}[1]{\mathbf{#1}}

\begin{document}

\title{Finite Element implementation of a Timoshenko beam for studing natural vibration modes in one-dimensional beams}
\date{July 25th, 2013}
\author{Manuel Diaz \\ manuel.ade@gmail.com \\ f99543083}

\maketitle

\begin{abstract}
A modal test is a form of vibration test of an object where the natural (modal) frequencies, modal masses and mode shapes of the object under test are determined. It's important is greatly recognized by industry due to many vibration phenomena encountered in engineering applications, specially in structural analysis and machine design \cite{Enboa_etal2002}. Timoshenko beam theory provides atractable way to study a structure modal response by providing a simple and reliable mathematical model \cite{Wiki}. By implementing finite element method (FEM) with Timoshenko beam theory it's possible to develop a simple lumped model that could help engineers understand and predict with great accuracy the natural frecuencies and vibration modes of a simple structural system. In this short article we present a detailed implementation of a Timoshenko beam model using FEM in Matlab to compute the modal responses of constant-section long \& short beams.
\end{abstract}

\section{Introduction}
The study of structural vibration through experiments has long provided a major contribution to understand and to control many vibration phenomena encountered in industry. It's important is easy identify during design and analysis stages of structures and goods. Typically problems are found in the service life of a structure as a result from vibration of structural systems. Typical examples are how to reduce weights of bridges, aircraft, machines and automobiles, etc. On the other hand, there are even more kinds of structural components or assemblies for which vibration is directly related to performance, either because of temporary malfunction during excessive motion or creating disturbance or discomfort. Therefore, it is important to develop tools that can help us to identify dangerous vibration levels encountered in service or operation.

Typically, two types of vibration test are frequenly used in industry. The first is normally done to measure the dynamic response of a structure or a machine by a pulse load. The second one is a test where a structure or a component is vibrated with controlled excitation, which is often not within the normal service environment. This second type of test generally yields more accurate and detailed information. This type of test is here called “Modal Testing” and is the subject of this experiment, which includes test arrangement, data acquisition and the subsequent analysis. Therefore, “Modal Testing” is defined here as “the processes involved in testing components or structures with the objective of obtaining a mathematical description of their dynamic or vibration behavior” \cite{Enboa_etal2002}.

\section{A (Simple) Modal Test Formulation}
For a discrete linear vibration system, the governing equation of motion is
\begin{align}
 \vec{M} \frac{\partial^2 \vec{x}}{\partial t^2} + \vec{C} \frac{\partial \vec{x}}{\partial t} + \vec{K} \vec{x} = \vec{F}
\label{eq:governing_equation}
\end{align}

where $\vec{x}$ is the displacement vector, $\vec{F}$ is the applied force vector, $\vec{M}$, $\vec{C}$ and $\vec{K}$ are the mass, damping and stiffness matrices, respectively. However for a simplified free vibration systems application we will neglet any contribution from a damping term. Then our governing equation would be,
\begin{align}
 \vec{M} \frac{\partial^2 \vec{x}}{\partial t^2} + \vec{K} \vec{x} = \vec{F}
\end{align}

Theoretically for this simple model, if we consider a system with free vibrations, $\vec{f}=0$, and assuming harmonic motion, 
\begin{align}
 u(t) = A sin(\omega t)
\end{align}
where $A$ is the vibration amplitude and $\omega$ the vibration frequencies. 
We now obtain the so called Generalized Eigen Problem.
\begin{align}
 [\vec{K}-\omega^2 \vec{M}]\vec{A} = \vec{0}
 \label{eq:Generalized_Eigen_Problem}
\end{align}
with no trivial solution by 
\begin{align}
 |\vec{K}-\omega^2 \vec{M}| = \vec{0}
\end{align}

For this simple case, as soon $\vec{K}$ and $\vec{M}$ matrices are known, we can solve the problem using a simple matlab implementation,

\lstinputlisting[firstline=73, lastline=76]{Timoshenko.m}

This simple code would lead to the solution of the principal modal frecuencies, $\omega_\lambda$, of our system.


In real practice the modal test is preformed by exiting a system by using a puntual load or by a mechanical resonator, and measuring the structure response with an accelerometer(s) attached to test them. Then using fourier decompositon we can analyze the phase of the mesasurements and determine the frecuencies at with the maximun amplitudes where registered. 

In some sence a modal test can be descibed as solving the same problem but starting with the frecuency values, and we wish to complete the picture but computing approximate contributios of the stiffness and mass matrices of the system under test. However structural theories come in handy by providing tools to predict a simple modal response or simple structures. Here is where Timoshenko beam theory can help us to develop a tractable way to deal with vibration related problems. 

\section{Implementation of Timoshenko Beam Theory}
Timoshenko Beam model in contrast to the Euler Beam model takes into account the deformation and rotation inertia effects, making it suitable for describing the behaviour of short beams, composite beams and beams subject to high-frecuency excitation when the wavelength approaches the thickness of the beam.

In order to implement Timoshenko beam theory into FEM and solve for equation \ref{eq:governing_equation},  it is evident that there are two requirements we need to fulfill before performing any modal analysis in a structure. We have to define a
\begin{itemize}
         \item Timoshenko Stiffness Matrix, and
         \item The systems Mass Matrix
         \item External Forces vector
\end{itemize}
For this derivation we follow ideas from \cite{Chakraverty2013,ferreira2007,Wiki}.

\subsection{Timoshenko's Stiffness Matrix}

In order to construct our stiffness matrix based on Timoshenko's beam theory as in the Bernoulli's beam theory. The first step is to define our kinematic, equilibrium and material relations

\begin{figure}
        \centering
        \begin{subfigure}[b]{0.48\textwidth}
                \centering
                \includegraphics[trim = 10mm 0mm 10mm 10mm,clip,width=\textwidth]{figures/TimoshenkoBeam}
                \caption{Comparison of beam theories.}
                \label{fig:Beam_comparison}
        \end{subfigure}%
        ~ %add desired spacing between images, e. g. ~, \quad, \qquad etc.
          %(or a blank line to force the subfigure onto a new line)
        \begin{subfigure}[b]{0.48\textwidth}
                \centering
                \includegraphics[trim = 5mm 5mm 5mm 5mm,clip,width=\textwidth]{figures/Timoshenko_Kinematics}
                \caption{Kinematic Relations.}
                \label{fig:Beam_kinematics}
        \end{subfigure}
        \caption{Figures for deriving kinematic relations}
	\label{fig:Timoshenko_kinematics}
\end{figure}

\begin{figure}
        \centering
        \includegraphics[trim = 20mm 80mm 22mm 70mm,clip,width=0.9\textwidth]{figures/Timoshenko_equilibrium}
        \caption{Figure for deriving equilibrium relations}
	\label{fig:Timoshenko_equilibrium}
\end{figure}

\emph{Kinematic Relations},
\begin{subequations}
	\begin{align}
	\theta &= \frac{dw}{dx}-\gamma \\
	\beta &= -\frac{d\theta}{dx}
	\end{align}
\end{subequations}

\emph{Equilibrium Relations},
\begin{subequations}
	\begin{align}
	q &= -\frac{dQ}{dx} = -Q' \\
	m &= -\frac{dM}{dx} + Q = -M' + Q
	\end{align}
\end{subequations}

\emph{Material Relations},
\begin{subequations}
	\begin{align}
	M = E I \beta \\
	Q = \alpha G A \gamma
	\end{align}
\end{subequations}
One of the most simple way to develop a FEM weak formulation for a Timoshenko beam in equilibrium is to use the virtual work principle, \cite{Wiki},
\begin{equation}
	\delta W - \delta U = 0
	\label{eq:virtual_work_principle}
\end{equation}
Here, $\delta W$ denotes the external virtual work while $\delta U$ denotes the internal virtual work. \\
From the kinematic Assumptions for a timoshenko bema, the displacements of the beam are given by
\begin{align}
	&u_x(x,y,z,t)=-z\theta(x,t);& &u_y(x,y,z,t)=0;& &u_z=w(x,t).&
\end{align}
Then, from the strain-displacement relations for small strains, the non-zero strains on the timoshenko assumptions are
\begin{align}
	&\epsilon_{xx} = \frac{\partial u_x}{\partial x};& 
	&\epsilon_{yy} = 0;& 
	&\gamma_{xy}   = \frac{\partial u_x}{\partial z}+\frac{\partial u_z}{\partial x} 
		= -\theta+\frac{\partial w}{\partial x}&
\end{align}
Since the actual shear strain in the beam is not constant over the cross section we introduce a correction factor $\kappa$ such that 
\begin{align}
	\gamma_{xy} = \kappa \left( -\theta + \frac{\partial w}{\partial x} \right)
\end{align}
The variation in the internal virtual energy of the beam is
\begin{align}
	\begin{split}
	\delta U &= \int_L \int_A (\sigma_xx\delta\epsilon_xx+\tau_xy\delta\epsilon_xy)dAdL \\
		&= \int_L \int_A \left[ -z \sigma_{xx}\frac{\partial \delta\theta}{\partial x}
		+ \tau_{xz} \kappa \left( -\theta + \frac{\partial w}{\partial x} \right) \right] dAdL 	
	\end{split}
	\label{eq:dev_internal_virtual_energy}
\end{align}
we now define
\begin{align}
	&M_{xx} = \int_A z \sigma_{xx} dA;& &Q_x = \kappa \int_A \tau_{xy} dA&
\end{align}
Using the kinematic relations we can substitute in \ref{eq:dev_internal_virtual_energy}, so we can rewrite in a much simpler form:
\begin{align}
	\delta U = \int_L (M_{xx} \delta \beta + Q_x \delta \gamma) dL
				 = \int_0^l (M_{xx} \delta \beta + Q_x \delta \gamma) dx
\end{align}
And the external virtual work can be written as,
\begin{align}
	\delta W = \int_0^l (\bar{m}\delta \beta +\bar{q}\delta \gamma) dx 
				+ \bar{M_0}\delta \beta_0 + \bar{M_l}\delta \beta_l
				+ \bar{Q_0}\delta \gamma_0 + \bar{B_l}\delta \gamma_l 
\end{align}
Substituting in to the virtual work principel, eq. (\ref{eq:virtual_work_principle}), we obtain the following relation
\begin{align}
	\begin{split}
	 \int_0^l (\bar{m}\delta \beta &- M_{xx} \delta \beta 
		+ \bar{q}\delta \gamma - Q_x \delta \gamma) dx \\
		&+ \bar{M_0}\delta \beta_0 + \bar{M_l}\delta \beta_l
		+ \bar{Q_0}\delta \gamma_0 + \bar{Q_l}\delta \gamma_l = 0
	\end{split}
	\label{eq:virtual_work_beam}
\end{align}

This last result can be writen in a more familiar manner by using the following relations:
\begin{align*}
	&\vec{\sigma} =
	\begin{bmatrix}
	Q \\
	M
	\end{bmatrix}&
	&\vec{\epsilon} =
	\begin{bmatrix}
	\gamma \\
	\beta
	\end{bmatrix}&
	&\vec{p} =
	\begin{bmatrix}
	\bar{q} \\
	\bar{m}
	\end{bmatrix}&
	&\vec{u} =
	\begin{bmatrix}
	w \\
	\theta
	\end{bmatrix}& 
\end{align*}
\begin{align*}
	&\vec{C} =
	\begin{bmatrix}
	\kappa G A	& 0 \\
	0				& EI
	\end{bmatrix}& 
	&\vec{L} = 
	\begin{bmatrix}
	\frac{\partial }{\partial x}	& 1\\
	0	& \frac{\partial }{\partial x}
	\end{bmatrix}
\end{align*}
Here matrix $\vec{C}$ will be called the constitutive matrix. Equation (\ref{eq:virtual_work_beam}) becomes,
\begin{align} 
	\int_0^l ( \vec{\sigma}^T \cdot \vec{\delta \epsilon} - \vec{p} \cdot \vec{\delta u} ) dx 
	- \bar{\vec{P}}_0^T \cdot \vec{\delta u}_0 - \bar{\vec{P}}_l^T \cdot \vec{\delta u}_l = 0
\end{align}
notice that $\vec{\sigma} = \vec{C} \cdot \vec{\epsilon}$ and $\vec{\epsilon} = \vec{L} \cdot \vec{u}$, subtituting in our last result yields
\begin{align}
	\int_0^l (\vec{u}^T \vec{L}^T \vec{C} \vec{L} \cdot \vec{\delta u} 
	- \vec{p} \cdot \vec{\delta u})dx 
	- [\bar{\vec{P}}_0 , \bar{\vec{P}}_l] \cdot \vec{\delta u} = 0
\end{align}
and assume and admisible solution can be expressed as $\vec{u} \approx \vec{u}_h = \vec{N} \cdot \vec{d}$,
\begin{align}
	\int_0^l (\vec{d}^T \vec{B}^T \vec{C} \vec{B} \cdot \vec{\delta d} 
	- \vec{p} \cdot \vec{N} \cdot \vec{\delta d})dx 
	- [\bar{\vec{P}}_0 , \bar{\vec{P}}_l] \cdot \vec{\delta d} = 0
\end{align}
Re-arraging the terms and simplifing common factor we obtain the our final FEM discrete formulation for a Timoshenko beam,
\begin{align}
	\int_0^l \vec{B}^T \vec{C} \vec{B} dx \cdot \vec{d} =
 	\int_0^l \vec{p} \cdot \vec{N} dx +
	\begin{bmatrix}
	 \bar{\vec{P}}_0 \\
	 \bar{\vec{P}}_l
	\end{bmatrix}	
\end{align}
Here we identify factor $\int_0^l \vec{B}^T \vec{C} \vec{B} dx \cdot \vec{d}$ as the system Strain Energy. Evaluating explicitly it would yields two the double product of the strain Energy,
\begin{align}
	2U = \int_0^l EI  \left( \frac{\partial \theta}{\partial x} \right)^2 dx
	+ \int_0^l \kappa AG \left( \frac{\partial w}{\partial x} - \theta \right)^2 dx  
\label{eq:strainEnergy}
\end{align}
The result show in equation (\ref{eq:strainEnergy}) is in good agreement with the energy contribution of strain due to bending and shear proposed by Ferreira \cite{ferreira2007}. Using this result we can follow with confidence the discretization proposed in \cite{ferreira2007}.\\

In opposition to to bernoulli beams, here the interpolation of displacements is independent for both $w$ and $\theta$,
\begin{align}
  \vec{w}  &= \vec{N} \vec{w}^e \\
  \vec{\theta}  &= \vec{N} \vec{\theta}^e
\end{align}

where shape functions are defined as
\begin{align}
  \vec{N} =
  \begin{bmatrix}
   \frac{1}{2}(1 - \xi) & \frac{1}{2}(1 + \xi)
  \end{bmatrix}
\end{align}

therefore in natural coordiantes $\xi \in [-1,+1]$. We can now compute the stiffness matrix as
\begin{align}
  \begin{split}
  \vec{K}^e &= \int_{-1}^{1} \frac{EI}{a^2} 
	\left( \frac{\partial N}{\partial \xi} \right)^T 
	\left( \frac{\partial N}{\partial \xi} \right) a d\xi \\
	  &+ \int_{-1}^{1} \kappa AG 
	\left( \frac{1}{a} \frac{\partial N}{\partial \xi} -N \right)^T 
	\left( \frac{1}{a} \frac{\partial N}{\partial \xi} -N \right) a d\xi
  \end{split}
\label{eq:stiffness_matrix}
\end{align}
where $a = Det[J]$. 
Notice that in (\ref{eq:stiffness_matrix}) is easy to identify the contribution of bending and shear terms separately. Furthermore, we notice that the bending contribution is missilar to what we have learned in \cite{Chen2013}, thus we only have to compute the shear contribution for the stiffness matrix.

Evaluating our stiffness matrix using the L2-isoparametric shape functions, we can write explicitly the stiffness matrix for each element,  

\begin{align}
 \vec{K}^e = \frac{EI}{2a} 
 \begin{bmatrix}
  1 & -1 \\
  -1 & 1
 \end{bmatrix} +
 \frac{\kappa GA}{6a}
 \begin{bmatrix}
  4a^2+6a+3 & 2a^2-3 \\
  2a^2-3 & 4a^2-6a+3
 \end{bmatrix}
\end{align}
We note that the integration of the bending contribution part numerically will only require one gauss point in contrast the numericall integration of the share contribution will require al least 2 gauss points.

\subsection{The Mass Matrix}
Following ideas from lectures notes in \cite{Chen2013},The mass matrix can be computed as follows:
\begin{align}
\vec{M}^e = \int_{-1}^{1} \rho A N^T N a d\xi
	  + \int_{-1}^{1} \rho I N^T N a d\xi
\label{eq:mass_matrix}
\end{align}
Similarly, if we use the L2-isoparametric shape function we can write explicitly the contribution of the mass and innertial parts,
\begin{align}
\vec{M}^e = \frac{\rho A a}{3}
 \begin{bmatrix}
  2 & 1 \\
  1 & 2 
 \end{bmatrix} +
 \frac{\rho I a}{3}
 \begin{bmatrix}
  2 & 1 \\
  1 & 2 
 \end{bmatrix}
\end{align}
The numerical integration of both contributions will require two gauss points.

\subsection{External Forces}
The work done by external forces, as we learned in \cite{Chen2013} is given by
\begin{align}
  \delta W^e &= \int_{-a}^{a} p \delta wdx = \int_{-1}^{1} p \delta w a d\xi \\
  \vec{f}^e &= \int_{-1}^{1} p N^T a d\xi
\end{align}
Using L2-isoparametric functions we can write explicitly this results for both the case of a constant distributed force,
\begin{align}
\vec{f}^e = 
 \begin{bmatrix}
  a P \\
  a P
 \end{bmatrix}
\end{align}
and for a non-uniformly distributed load using interpolation,
\begin{align}
\vec{f}^e = \int_{-1}^{1} p [N]^T [N] [p_1,p_2]^T a d\xi = 
 \begin{bmatrix}
  \frac{a}{3} (2p_1+p_2) \\
  \frac{a}{3} (p_1+2p_2)
 \end{bmatrix}
\end{align}


\subsection{Boundary Conditions}
The boundary conditions for our case can be expresed as \\
\emph{Essential/Dirichlet}
\begin{align*}
	&w(0)=w_0&  &w(l)=w_l& \\
	&\theta(0)=\theta_0 & &\theta(0)=\theta_l &
\end{align*}
\emph{Natural/Neumann}
\begin{align*}
	&Q(0)=\bar{Q}_0& &Q(l)=-\bar{Q}_l & \\
	&M(0)=\bar{M}_0& &Q(l)=-\bar{M}_l &
\end{align*}

\section{Matlab Implementation}
In this section we start describing step-by-step the finite element implementation of Timoshenko beam theory to make a modal analysis in a one-dimensional constant-section beam.

\subsection{Step 1: Define initial conditions}
As we learned with \cite{Chen2013}, we start our implementation defining the material constants and geometrical characteristics of our model: Young Modulus, $E$; Beam Lenght, $L$; Beam Thickness; Cross Section Innertia respect to z-axis, $I$; shear correction factor, $\kappa$; Material Density, $rho$ and section area, $A$.

\lstinputlisting[firstline=11, lastline=21]{Timoshenko.m}

We also define before hand the modulus of rigidity, G, and the constitutive matrix, $\vec{C}$,

\lstinputlisting[firstline=25, lastline=27]{Timoshenko.m}

\subsection{Step 2: Define beam's DOF's and Essential BC's}

In step we define the number of elements, an array with each elements nodes coordiantes, a node connectivity array and estimate the total number of degress of freedom of our system.

\lstinputlisting[firstline=29, lastline=41]{Timoshenko.m}

\subsection{Step 3: Construct Stiffness, Mass Matrix and external force vector}
The computation of stiffness and Mass Matrix will be taken a part as the function: formStiffnessMassTimoshenkoBeam.m fucntion. The codes that follow depict the numerical implementation of equations: (26), (28) and (31).

\subsubsection{Bendind contribution, Mass matrix and Force vector}
Similarly to what we learn in \cite{Chen2013},

\lstinputlisting[firstline=10, lastline=38]{formStiffnessMassTimoshenkoBeam.m}

\subsubsection{Shear contribution}
Implementation for the shear term in the Stiffness Matrix definition,

\lstinputlisting[firstline=39, lastline=61]{formStiffnessMassTimoshenkoBeam.m}

Notice in lines 15-17 the redefinition of the B factor to take into account the Xderivate of the shape function and the shape itself to comply with the definition in the last term of equation (\ref{eq:stiffness_matrix}). 

\subsection{step 4: Setup Essential Boundary Conditions}
As we learnd with in \cite{Chen2013}, when using FEM we only have to comply with the essential boundary conditions of our system. For a one-dimensional beam we can setup up to four kind of boundary conditions, \\

\emph{Simply-supported at both bords:} 
\lstinputlisting[firstline=49, lastline=50]{Timoshenko.m}

\emph{Clamped at both bords:} 
\lstinputlisting[firstline=52, lastline=53]{Timoshenko.m}

\emph{Cantilever:} 
\lstinputlisting[firstline=55, lastline=56]{Timoshenko.m}

\emph{Free-free:} 
\lstinputlisting[firstline=58, lastline=59]{Timoshenko.m}

\subsection{step 5*: Solve for the displacements and Reactions}
Once the Timoshenko stiffness matrix is defined we can solve the beam for the displacements and reactions in the traditional way learn in \cite{Chen2013}.

\lstinputlisting[firstline=62, lastline=67]{Timoshenko.m}

\subsection{step 5: Solve for the Modal frecuencies}
As we exemplified in section 2, once the stiffness and mass matrix are know we can solve the generalized eigenvalue problem, equation (\ref{eq:Generalized_Eigen_Problem}), to solve for the modal frequencies of our system. This is implemented in our code in the following lines:

\lstinputlisting[firstline=69, lastline=77]{Timoshenko.m}

Here $D$ is the eigenvalues diagonal matrix, which is simplified and sorted so that the final results are the modal frecuencies in given in Hertz units [Hz], listed in magnitude from lower to higher.

\subsection{step 6: Plot Results}
In this section we employ the subroutine found in Chp. 10 of \cite{ferreira2007} to draw the mode shapes in our beam.

\lstinputlisting[firstline=81, lastline=82]{Timoshenko.m}

\section{Numerical Results}

\subsection{Evaluation modal shapes of BCs}
Here we evaluate the code's output and plot the resulting modal shapes for different configuration of boundary conditions. Figure \ref{fig:Mode_Shapes} show the results on the proposed BC's in section 4.4. 

The graphical results show qualitatively a good agreement to the expected behavior in every configuration. Furthermore, we have observed that the response is only influence by the geometrical and material characteristics of the beam, as expected.

\begin{figure}
        \centering
        \begin{subfigure}[b]{0.48\textwidth}
                \centering
                \includegraphics[trim = 20mm 40mm 20mm 5mm,clip,width=0.9\textwidth]{figures/cantilever}
		\caption{Cantilever BC's}
		\label{fig:Cantilever}
        \end{subfigure}
        ~ %add desired spacing between images, e. g. ~, \quad, \qquad etc.
          %(or a blank line to force the subfigure onto a new line)
        \begin{subfigure}[b]{0.48\textwidth}
                \centering
                \includegraphics[trim = 20mm 40mm 20mm 5mm,clip,width=0.9\textwidth]{figures/free-free}
		\caption{Free-free BC's}
		\label{fig:Free-free}
        \end{subfigure}
        ~ %add desired spacing between images, e. g. ~, \quad, \qquad etc.
          %(or a blank line to force the subfigure onto a new line)
        \begin{subfigure}[b]{0.48\textwidth}
                \centering
                \includegraphics[trim = 20mm 40mm 20mm 5mm,clip,width=0.9\textwidth]{figures/simply_supported_both_sides}
		\caption{Simply-supported at both sides}
		\label{fig:Simply_supported_both_sides}
        \end{subfigure}%
        ~ %add desired spacing between images, e. g. ~, \quad, \qquad etc.
          %(or a blank line to force the subfigure onto a new line)
        \begin{subfigure}[b]{0.48\textwidth}
                \centering
                \includegraphics[trim = 20mm 40mm 20mm 5mm,clip,width=0.9\textwidth]{figures/clamped_both_sides}
		\caption{Clamped at both sides}
		\label{fig:Clamped_both_sides}
        \end{subfigure}
        \caption{First 4 mode shapes for proposed BC's using 50 elements.}
	\label{fig:Mode_Shapes}
\end{figure}

\subsection{Comparison of modal frequencies}
Here we choose the solutions of our clamped model and proceed to test if the model converges to the classical solution of it's respective configuration. 
\begin{center}
  \begin{tabular}{crrrrrc}
  \hline
  \multicolumn{7}{c}{Solutions by Finite Elements} \\
  \cline{2-6}
  Mode & 1 elem.  & 2 elem. & 5 elem. & 10 elem. & 50 elem. & Exact Solution \\
  \hline
  1    & 3.4639   & 3.5915  & 3.5321  & 3.5200   & 3.5159   & 3.516  \\
  2    & 58.8390  & 40.3495 & 24.2972 & 22.5703  & 22.0439  & 22.035 \\
  \hline
  \end{tabular}
\end{center}
We observe again a good agreement on the approximation using FEM.

\section{Conclusions and Observations}
\begin{itemize}
 \item Timoshenko beams theory is high versatil for dealing with any size of beam, however it's complete theory still under develop as we can find so many good articles related to this topic, specially in the range on nano applications like \cite{Chakraverty2013}.
 \item Future developments in Timoshenko beam theory is getting towards mindlin plates (basically two-dimensional plates) targeting applications in resonating micro and nanostrutures.
 \item It's interesting to notice, (and the author wishes to remark) how easy is to apply virtual work principles and energy principles to obtain and/or construct FEM's weak formulations for Timoshenko and Euler beam theories. Further details of this remark can be found by traking actual and previous work by \cite{ferreira2007}.
 \item When using Timoshenko beam theory, because of vertical displacements are not directly approximated with the slope rotation of the section plane, we are not restrictted by complying with shape $C^1$ functions, e.i. ensuring continuity up to second order derivate. Therefore we can use $C^0$ functions to formulate an approximate solution with Timoshenko beam formulation.
 \item In the present article we used L2-isoparametric shape functions, however researchers are aware of the lock disadvantages of using Timoshenko theory with linear function. The general recomendation is to use higher shape function to avoid any locking phenomena.  
\end{itemize}

\begin{thebibliography}{9}

\bibitem{Chen2013}
	Chen David, 
	\emph{Lecture Notes on Finite Element Methods}
	Chaptes 1-6,
	National Taiwan University,
	Spring 2013.

\bibitem{Chakraverty2013}
	Laxmi Behera \& S. Chakraverty,
	\emph{Free vibration of nonhomogenous Timoshenko nanobeams}
	Meccanica Journal,
	June 2013.
	DOI:10.1007/s11012-013-9771-2

\bibitem{ferreira2007}
	Ferreira A.J.M,
	\emph{Matlab Codes for Finite Elements Analysis}
	Ch. 10,
	Springer,
	2007.

\bibitem{Enboa_etal2002}
	Wu Enboa, et al.;
	\emph{Modal Testing}
	Experiments in Applied Mechanics,
	Institute of Applied Mechanics,
	National Taiwan University,
	2002.

\bibitem{Wiki}
	Wikipedia.org,
	\emph{Timoshenko Beam Theory}
	Available on: http://en.wikipedia.org/wiki/Timoshenko\_beam\_theory,
	Accessed on: 2013.07.20.

	
\end{thebibliography}

% This is the end of the Article
\end{document}
