\documentclass[a4paper]{memoir}

% Preamble
\usepackage{calc}
\usepackage{color}
\usepackage{graphicx}
\usepackage{amsmath}
\usepackage{amssymb}

%define title and other basic document info
%the title should reflect the style and give a foretaste of the document
%work on making a stylized title page — or title on a page as in ARTICLE class
\title{\huge \textbf{Homework Problems No.2}}
\author{Manuel A. Diaz \\ f99543083}
\date{March 12th, 2013} % could use \today
%\publisher{Institute of Applied Mechanics, IAM}  %one day I’ll need this  
%\thanks{Special thanks to God for the ability to work}        %produces a footnote to the title

\definecolor{shadecolor}{gray}{0.9}
\definecolor{ared}{rgb}{.647,.129,.149}
\renewcommand\colorchapnum{\color{ared}}
\renewcommand\colorchaptitle{\color{ared}}
\chapterstyle{bringhurst}
%one of a number of chapter styles available…this one doesn’t use the ared color

% Begin Document Here
\begin{document}

% Build costumized Title Page
\thispagestyle{empty}
%\begin{minipage}{300pt}
\begin{center}{
\begin{shaded}
\hrule \vspace{30pt}
\hspace{30pt} \thetitle  \vspace{30pt}
\newline \theauthor \hspace{30pt} \thedate  \vspace{26pt}
\hrule
\end{shaded}
}
\end{center}
%\end{minipage}
\clearpage

%\frontmatter    %use if needed –page numbers as lower case roman numerals i, ii,…

%\mainmatter
%%other declarations
\pagestyle{Ruled}                    %one of a number of possible page styles
\midsloppy                             %to minimize overfull lines

%Layout the page
%%Try this manual golden ratio layout or…           default seems better for now
%\settypeblocksize{*}{\lxvchars}{1.618}
%\setulmargins{50pt}{*}{*}
%\setlrmargins{*}{*}{1.618}
%\setheaderspaces{*}{*}{1.618}
%\semiisopage[12]
%try this predefined layout — others predefined ones are options in MEMOIR…
%this one looked best but did not work

\checkandfixthelayout          %make the layout happen and provide details in log during build

\chapter{Homework Problems No. 2}
\section{Problem 1}
\subsection{Given}
Given the Example 3 in Chapter 1 of the lecture notes. See figure \ref{fig:uniform_cantilever}.

\begin{figure}
	\centering
		\includegraphics[width=0.5\textwidth]{uniform_cantilever.png}
	\caption{Example 3, uniform section cantilever}
	\label{fig:uniform_cantilever}
\end{figure}

\subsection{Find}
\begin{enumerate}
	\item The solution from a cubic polynomial $x(u) = a_0+a_1x+a_2x^2+a_3x^3$ using the Rayleigh-Ritz method.
	\item Compute and tabulate the force boundary conditions at x=L from the trial solution obtained from linear, quadratic and cubic polynomails and give a brief comment on your observation.
\end{enumerate}

\subsection{Plan}
From chapter 1 in lecture notes, we wish to find a smooth function $u(x)$ that satisfy $u = \bar{u}$ on $\Gamma_u$ such that

\begin{align*}
	&\int_\Omega \frac{dw}{dx} AE \frac{du}{dx} dx = (wA\bar{t})|_{\Gamma_t} + \int_\Omega wb dx,  & w = 0 \;\; on \;\; \Gamma_u
\end{align*}

Using Rayleigh-Ritz Method, the total energy for a 1D ideal elastic bar is:
\begin{equation}
	\Pi(u(x)) = \frac{1}{2} \int_\Omega AE \left( \frac{du}{dx} \right)^2 dx-\left( (uA\bar{t})|_{\Gamma_t} + \int_\Omega ub dx \right)
	\label{eq:cantilever_prob_eq}
\end{equation}
And that teh variation of the functional must vanish, just as the differentials of a function vanish at a minimun of a function. \\
For the fixed cantilever show in figure \ref{fig:uniform_cantilever}, the total potentail energy is:
\begin{equation}
	\Pi = \frac{1}{2} EA \int^L_0 \left( \frac{du}{dx} \right)^2 dx - c \int^L_0 xudx - Pu(L)
	\label{eq:ubeam_potencial_energy}
\end{equation}
As requested, we assume the solution to be a cubic polynomial
\begin{align*}
	 & u(x) = a_0+a_1x+a_2x^2+a_3x^3,\\ & \Rightarrow \frac{du}{dx} = a_1+2a_2x+3a_3x^2, \\
	 & \Rightarrow \left( \frac{du}{dx} \right)^2 = (a_1^2+4a_1a_2+6a_1a_3x^2+4a_2^2x^2+6a_2a_3x^3+9a_3^2x^4).
\end{align*}
however due to our fixed BC.
\begin{align*}
	 u(0) &= 0, \Rightarrow  a_0 = 0
\end{align*}

\subsection{Calculations}
Substituting our assumed cubic solution in to equation (\ref{eq:ubeam_potencial_energy}), we 
\begin{align*}
	&
	\begin{split}
	\Pi &= \frac{1}{2} EA \int^L_0 (a_1^2+4a_1a_2+6a_1a_3x^2+4a_2^2x^2+6a_2a_3x^3+9a_3^2x^4)dx \\ 
			&- c \int^L_0 x(a_1x+a_2x^2+a_3x^3)dx - P(a_1L+a_2L^2+a_3L^3)
	\end{split}
\end{align*}

Integrating,
\begin{align*}
	&
	\begin{split}
	\Pi &= \frac{1}{2} EA (a_1^2L+2a_1a_2L^2+2a_1a_3L^3+4/3a_2^2L^3+3a_2a_3L^4+9/5a_3^2L^5) \\
			&- c(\frac{1}{2}a_1L^2+\frac{2}{3}a_2L^3+\frac{3}{4}a_3L^5) - P(a_1L+a_2L^2+a_3L^3)
	\end{split} 
\end{align*} 
Taking the derivates respect to $a_i$ where $i = 1,2,3$, we get
\begin{align*}
	\frac{\partial \Pi}{\partial a_1} &=	\frac{1}{2} EA (2a_1L+2a_2L^2+2a_3L^3) - \frac{CL^3}{3} - PL = 0 \\
	\frac{\partial \Pi}{\partial a_2} &=	\frac{1}{2} EA (2a_1L^2+8/3a_2L^3+3a_3L^4) - \frac{CL^4}{4} - PL^2 = 0 \\
	\frac{\partial \Pi}{\partial a_3} &=	\frac{1}{2} EA (2a_1L^3+3a_2L^4+18/5a_3L^5) - \frac{CL^5}{5} - PL^3 = 0 
\end{align*}
		
simplyfing constants in the above equations: 
\begin{align*}
	&\frac{\partial \Pi}{da_1} = AE(La_1+L^2a_2+L^3a_3) - \frac{cL^3}{3} - LP = 0 \\
	&\frac{\partial \Pi}{da_1} = AE(La_1+4/3L^2a_2+3/2L^3a_3) - \frac{cL^4}{4} - LP = 0 \\
	&\frac{\partial \Pi}{da_1} = AE(La_1+3/2L^2a_2+9/5L^3a_3) - \frac{cL^5}{5} - LP = 0 
\end{align*}

it can be re-write in the following system of equation,
\begin{align*}
	&a_1+La_2+L^2a_3= \frac{1}{3AE} (cL^2+3P) \\
	&a_1+4/3La_2+3/2L^2a_3 = \frac{1}{4AE} (cL^2-4P) \\
 	&a_1+3/2L^2a_2+9/5L^3a_3 = \frac{1}{5AE} (cL^2-5P)
\end{align*}

Solving the above system of equations for $a_1$, $a_2$ and $a_3$ we find,
\begin{align*}
	&a_1 = \frac{cL^2-2P}{2AE}, &a_2 &= 0, &a_3 = -\frac{c}{6AE}
\end{align*}

Substituting into the assumed cubic polynomail, we get the following solution for the problem,

\subsection{Solution}
\begin{minipage}{300pt}
	\begin{center}{
		\begin{shaded}
			\hrule
			\vspace{20pt}
			Our quadratic solution function is: $u(x)$ = $\frac{cL^2+2P}{2EA}x-\frac{c}{6AE}x^3$ %nicely written sentence solution goes here
			\vspace{16pt}
			\hrule
		\end{shaded}
	}
	\end{center}
\end{minipage}

\subsection{Tabulating and comparing}
Now, tabulating the force boundary condition for $x=L$,

\begin{center}
	\begin{tabular*}{0.98\textwidth}{|c|c|c|}
	\hline
		Solution polynomial & $u(x)$ & $F|_{x=L} = AE\frac{du}{dx}|_{x=L}$ \\
	\hline
		Linear 		& $u(x)$ = $\frac{cL^3+3P}{3EA}x$	&	 $f(L) = \frac{cl^2 + 3P}{3}$ \\
	\hline
		Quadratic & $u(x)$ = $\frac{7cL^2+12P}{12EA}x-\frac{cl}{4AE}x^3$ &	$f(L) = \frac{cl^2 + 12P}{12}$ \\
	\hline
		Cubic			& $u(x)$ = $\frac{cL^2+2P}{2EA}x-\frac{c}{6AE}x^3$ & $f(L) = P$ \\
	\hline
	\end{tabular*}
\end{center}

We observe, that as the solution increses poolynomial degree the description of the force acting in the free side ($x=L$) of our beam becomes more and more accuarate. Finally we notice that the force becomes the exact input force value $P$ for the cubic solution. 


\section{Problem 2}
\subsection{Given}
The tapered bar fixed at one end and subject to a static point load at the other end as show in figure ref{fig:tappered cantilever}.
The bar is subject to a linearly varying axial load $q=cx$, where $c$ is a given constant. The area varies linearly from $A_0$ to $A_L$ where,
\begin{align*}
	&A(x) = \frac{(l+(-1+r)x)A_0}{L}	& r = A_l/A_0
\end{align*}

\begin{figure}
	\centering
		\includegraphics[width=0.5\textwidth]{tappered_cantilever.png}
	\caption{Tapered cantilever}
	\label{fig:tappered_cantilever}
\end{figure}

\subsection{Find}
Obtain the solution from a linear polynimal $u(x) = a_0+a_1x$ using Rayleight-Ritz method 

\subsection{Plan}
recall equation (\ref{eq:cantilever_prob_eq}),
\begin{align*}
	&\Pi = \frac{1}{2} \int_\Omega AE \left( \frac{du}{dx} \right)^2dx - (uA\bar{t})|_{\gamma_i} - \int_\Omega ubdx
\end{align*}
here we consider,
\begin{align*}
	& b = cx, & A(x) & = \frac{(1+(r-1)x)A_0}{L}, & r = \frac{A_L}{A_0}
\end{align*}

Assuming a linear solution of the form: 
\begin{align*}
	& u(x) = a_0+a_1x, \rightarrow \frac{du}{dx} = a_1 \\
	& \text{with,} \\
	& u(0) = 0, 	\rightarrow a_0 = 0
\end{align*}
 
Re-writting the expresion A(x) into a much simpler form:
\begin{align*}
	& A(x) = A_0 + (A_L-A_0)\frac{x}{L}
\end{align*}

Based on eq. (\ref{eq:cantilever_prob_eq}) and knowing that $A = A(x)$ the governing equation for our problem becomes,
\begin{equation}
	\Pi = \frac{1}{2} E \int^L_0 A(x) \left( \frac{du}{dx} \right)^2dx - c\int^L_0 u(x)xdx - Pu(L)
	\label{eq:tapered_problem_eq}
\end{equation}

\subsection{Calculations}

substitution into equation (\ref{eq:tapered_problem_eq}),
\begin{align*}
	\Pi &= \frac{1}{2} E \int^L_0 (A_0+(A_L-A_0)\frac{x}{L})(a_1)^2 dx - c\int^L_0 (a_1x)xdx - P(a_1L) \\	
	& \text{Simplifying and integrating,} \\
	\Pi &= \frac{1}{2} E \int^L_0 (A_0a_1^2L+\frac{(A_L-A_0)a_1^2}{L})(a_1)^2 dx - \frac{1}{3}cL^3a_1 - PLa_1
\end{align*}

Deriving the above potential equation with respect to $a_1$,
\begin{align*}
	&\frac{\partial \Pi}{\partial a_1} = \frac{1}{2}E (2A_0La_1 + (A_L-A_0)L^2a_1) - \frac{1}{3}cL^3 - PL =0 \\
   & \text{solving for $a_1$ we get,} \\
   & a_1 = \frac{CL^2+3P}{3E(A_0+1/2(A_L-A_0))} = \frac{2(CL^2+3P)}{3E(A_L+A_0)}  
\end{align*}

\subsection{Solution}
\begin{minipage}{300pt}
	\begin{center}{
		\begin{shaded}
			\hrule
			\vspace{20pt}
			Our linear solution is, $u(x) = \frac{2(CL^2+3P)}{3E(A_L+A_0)} x$  %nicely written sentence solution goes here
			\vspace{16pt}
			\hrule
		\end{shaded}
	}
	\end{center}
\end{minipage}

\section{Bonus Problem}
\subsection{Given}
Consider again the tapered bar fixed at one end and subject to a static point load in problem 2.

\subsection{Find}
Continue the derivation from problem 2 and obtain the solution from a quadratic polynomial $x(u) = a_0+a_1x+a_2x^2$ using Rayleigh-Ritz method.

\subsection{Plan}
Consider again eq. (\ref{eq:tapered_problem_eq}) 
\begin{equation}
	\Pi = \frac{1}{2} E \int^L_0 A(x) \left( \frac{du}{dx} \right)^2dx - c\int^L_0 u(x)xdx - Pu(L)
\end{equation}

where
\begin{align*}
	& b = cx, & A(x) &= \frac{(1+(r-1)x)A_0}{L}, & r = \frac{A_L}{A_0}
\end{align*}

assuming a linear solution of the form: 
\begin{align*}
	& u(x) = a_0+a_1x+a_2x^2, \rightarrow \frac{du}{dx} = a_1+2a_2x \\
	& \left( \frac{du}{dx} \right)^2 = a_1^2+4a_1a_2x+4a_2x^2 \\
	& \text{with,} \\
	& u(0) = 0, 	\rightarrow a_0 = 0
\end{align*}
 
re-writting the expresion A(x) into a much simpler form:
\begin{align*}
	& A(x) = A_0 + (A_L-A_0)\frac{x}{L}
\end{align*}

\subsection{Calculations}

substitution into equation (\ref{eq:tapered_problem_eq}),
\begin{align*}
	\begin{split}
	\Pi &= \frac{1}{2} E \int^L_0 (A_0+(A_L-A_0)\frac{x}{L})(a_1^2+4a_1a_2x+4a_2x^2) dx \\
			&- c\int^L_0 (a_1x+a_2x^2)xdx - P(a_1L+a_2L^2)	
	\end{split}
\end{align*}
Integrating the potential equation,
\begin{align*}
	\begin{split}
	\Pi &= \frac{1}{2}E( A_0a_1^2L + 1/2(A_L-A_0)a_1^2L + 4/2A_0a_1a_2L^2 + 4/3(A_L-A_0)a_1a_2L^2 \\
			&+ 4/3A_0a_2^2L^3 + (A_L-A_0)a_2^2L^4) - c(\frac{1}{3}a_1L^3+\frac{1}{4}a_2L^3) - P(a_1L+a_2L^2)
	\end{split}
\end{align*}
simplifying the above equation,
\begin{align*}
	&
	\begin{split}
	\Pi &= \frac{1}{2}E((A_0+1/2(A_L-A_0))a_1^2L + 4(1/2A_0+1/3(A_L-A_0))a_1a_2L^2 \\
			&+ 4(1/3A_0+1/4(A_L-A_0)a_2^2L^3)) - c\frac{1}{3}a_1L^3+c\frac{1}{4}a_2L^4 - a_1PL+a_2PL^2
	\end{split} 
\end{align*}

\begin{align*}
	&
	\begin{split}
	\Pi &= \frac{1}{2}E( 1/2(A_0+A_L)a_1^2L + 4/6(A_0+2A_L)a_1a_2L^2 \\
			&+ 4/12(A_0-3A_L)a_2^2L^4) - c\frac{1}{3}a_1L^3+c\frac{1}{4}a_2L^4 - a_1PL+a_2PL^2
	\end{split}
\end{align*}

\begin{align*}
	&
	\begin{split}
	\Pi &= \frac{1}{12}EL( 3(A_0+A_L)a_1^2 + 4(A_0+2A_L)a_1a_2L \\
			&+ 2(A_0-3A_L)a_2^2L^2) - c\frac{1}{3}a_1L^3+c\frac{1}{4}a_2L^4 - a_1PL+a_2PL^2
	\end{split}
\end{align*}

Taking the derivates of the potential equation with respect to $a_i$ where $i = 1,2,3$, we get
\begin{align*}
	\frac{\partial \Pi}{\partial a_1} &= \frac{1}{12} EL (6a_1(A_0+A_L)+4a_2L(A_0+2A_L)) - \frac{CL^3}{3} - PL = 0 \\
	\frac{\partial \Pi}{\partial a_2} &= \frac{1}{12} EL (4a_1(A_0+2A_L)+4a_2L^2(A_0+3A_L)) - \frac{CL^4}{4} - PL^2 = 0
\end{align*}

Re-arraging the equations above,
\begin{align*}
	 \frac{1}{12} EL (6a_1(A_0+A_L)+4a_2L(A_0+2A_L)) &= \frac{CL^3}{3} + PL  \\
	 \frac{1}{12} EL (4a_1L(A_0+2A_L)+4a_2L^2(A_0+3A_L)) &= \frac{CL^4}{4} + PL^2 
\end{align*}

Solving the above system of equations for $a_1$ and $a_2$ we find,
\begin{align*}
	&a_1= -\frac{cL^2A_0+6cL^2A_L+12PA_L}{2E(A_0^2+4A_0A_L+A_L^2)}, 
	&a_2= \frac{cL^2A_0+12PA_0-7cL^2A_L+12PA_L}{4E(A_0^2+4A_0A_L+A_L^2)}
\end{align*}

\subsection{Solution}
\begin{minipage}{300pt}
	\begin{center}{
		\begin{shaded}
			\hrule
			\vspace{20pt}
			  Our quadratic solution is then: %nicely written sentence solution goes here
				\begin{align*}
					u(x)= \frac{(cL^2A_0+12PA_0-7cL^2A_L+12PA_L)x^3-2(cL^2A_0+6cL^2A_L+12PA_L)x}{4E(A_0^2+4A_0A_L+A_L^2)}
				\end{align*}
			\vspace{16pt}
			\hrule
		\end{shaded}
	}
	\end{center}
\end{minipage}

\end{document}