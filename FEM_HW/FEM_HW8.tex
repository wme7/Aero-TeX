\documentclass[a4paper]{memoir}

% Preamble
\usepackage{calc}
\usepackage{color}
\usepackage{graphicx}
\usepackage{amsmath}
\usepackage{amssymb}

%define title and other basic document info
%the title should reflect the style and give a foretaste of the document
%work on making a stylized title page — or title on a page as in ARTICLE class
\title{\huge \textbf{Homework Problems No.8}}
\author{Manuel A. Diaz \\ f99543083}
\date{May 21st, 2013} % could use \today
%\publisher{Institute of Applied Mechanics, IAM}  %one day I’ll need this  
%\thanks{Special thanks to God for the ability to work}        %produces a footnote to the title

\definecolor{shadecolor}{gray}{0.9}
\definecolor{ared}{rgb}{.647,.129,.149}
\renewcommand\colorchapnum{\color{ared}}
\renewcommand\colorchaptitle{\color{ared}}
\chapterstyle{bringhurst}
%one of a number of chapter styles available…this one doesn’t use the ared color

% Begin Document Here
\begin{document}

% Build costumized Title Page
\thispagestyle{empty}
%\begin{minipage}{300pt}
\begin{center}{
\begin{shaded}
\hrule \vspace{30pt}
\hspace{30pt} \thetitle  \vspace{30pt}
\newline \theauthor \hspace{30pt} \thedate  \vspace{26pt}
\hrule
\end{shaded}
}
\end{center}
%\end{minipage}
\clearpage

%\frontmatter    %use if needed –page numbers as lower case roman numerals i, ii,…

%\mainmatter
%%other declarations
\pagestyle{Ruled}                    %one of a number of possible page styles
\midsloppy                             %to minimize overfull lines

%Layout the page
%%Try this manual golden ratio layout or…           default seems better for now
%\settypeblocksize{*}{\lxvchars}{1.618}
%\setulmargins{50pt}{*}{*}
%\setlrmargins{*}{*}{1.618}
%\setheaderspaces{*}{*}{1.618}
%\semiisopage[12]
%try this predefined layout — others predefined ones are options in MEMOIR…
%this one looked best but did not work

\checkandfixthelayout          %make the layout happen and provide details in log during build


\chapter{Homework Problems No. 6}
\section{Problem 1}
\subsection{Given}
A 12-node serendipity type element depicted in figure \ref{fig:element1},

\subsection{Derive}
Derive $N_2$ and $N_9$ shape functions for the element.

\begin{figure}
	\centering
		\includegraphics[width=0.35\textwidth]{Element1.png}
	\caption{Problem 1, 12-node Serenditipy quadrilateral type element}
	\label{fig:element1}
\end{figure}

%\subsection{Plan}

%\subsection{Calculations}

%\subsection{Solution}

\section{Problem 2}
\subsection{Given}
A 6-node serendipity type element depicted in figure \ref{fig:element2},

\subsection{Derive}
Derive $N_1$, $N_5$ and $N_6$ shape functions for the element.

\begin{figure}
	\centering
		\includegraphics[width=0.3\textwidth]{Element2.png}
	\caption{Problem 2, 6-nodes Serenditipy quadrilateral type element}
	\label{fig:element2}
\end{figure}

%\subsection{Plan}

%\subsection{Calculations}

%\subsection{Solution}


\end{document}